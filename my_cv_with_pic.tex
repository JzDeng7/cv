\documentclass[12pt]{article}

\usepackage{myCv}

\geometry{a4paper, scale=0.8}

\begin{document}
    % 个人基本信息
    \newPartBar{\faUser}{基本信息}
    % 个人基本信息的内容
    \vspace{2mm}
    \begin{figure}[H]
    \begin{varwidth}[t]{0.5\textwidth}
    	\vspace{0pt}
        \baseInfoName{8}{0.4}{邓竣泽}\baseInfoSex{0}{0.4}{男}
        \baseInfoAge{8}{0.4}{21}\baseInfoNation{0}{0.4}{汉}
        \baseInfoHome{8}{0.4}{湖南湘潭}
        \baseInfoSchool{8}{0.8}{四川大学}
        \baseInfoMajor{8}{0.8}{物理学(基地班)}
    \end{varwidth}
    % 照片
	\addPic{30}{0.1}{pic/IMG_1430.png}

    \end{figure}

    % 学习Bar
    \newPartBar{\faPencil}{学习情况}
    % 学习情况内容
    \vspace{5mm}
    % \begin{spacing}{1}
    \begin{varwidth}[t]{0.8\textwidth}
        \infoItem{0}{0.4}{加权成绩}{85.36/100}\infoItem{1}{0.5}{GPA}{3.40/4}
        \infoItem{0}{0.4}{专业排名}{10/86}
        \infoItem{0}{0.6}{六级成绩}{526}
        \infoItem{0}{0.6}{托福成绩}{79}   
    \end{varwidth}
    % \end{spacing}
    \vspace{5mm}
    
    
    \newPartBar{\faGift}{主要科目成绩}
    \vspace{5mm}
    \begin{varwidth}[t]{0.8\textwidth}
    \infoItem{0}{0.4}{理论力学}{91} \infoItem{0}{0.4}{电动力学}{79}
    \infoItem{0}{0.4}{统计物理}{81} \infoItem{0}{0.4}{量子力学}{88}
    \infoItem{0}{0.4}{固体物理}{85} \infoItem{0}{0.4}{量子场论}{89}
    \infoItem{0}{0.4}{计算物理}{94} \infoItem{0}{0.4}{数学物理方法}{86}
	\end{varwidth}
        
    \vspace{5mm}
    % 项目和科研情况
    \newPartBar{\faCog}{科研经历}
    % 项目和科研内容
    \vspace{2mm}
    \begin{myCvItems}
            \item 大学生创新创业项目“磁性纳米材料”:评估了微波加热法和溶胶凝胶法制备二维纳米材料的可行性,计算了在掺杂了Fe后GeSe薄膜的物理性质 \hfill 2017-2018
            \item 大学生创新创业项目(本科毕业设计课题)“两体系统在有限体积中的分立能级”:利用格点计算背景中的立方对称性简化中子散射的实验数据计算\hfill2018-2019 
    \end{myCvItems}

    % 获奖和证书情况
    \newPartBar{\faStar}{获奖情况}
    % 获奖和证书内容
    \vspace{3mm}
    \begin{myCvItems}
            \item 中国科学院物理研究所“严济慈班”奖学金\hfill 2015-2017
            \item 四川大学综合二等奖学金\hfill 2017-2018  
            \item 四川大学综合三等奖学金\hfill 2016-2017
            \item 四川大学数学竞赛三等奖\hfill 2016.6 
    \end{myCvItems}

    % ---
    % 个人技能
    %\newPartBar{\faGift}{个人技能}
    % 个人技能内容
    %    \begin{myCvItems}
    %        \item 较为熟练地吃饭。
    %        \item 较为熟练地利用\LaTeX 进行菜谱编写。
    %        \item 较为熟练地使用勺子吃饭。
    %    \end{myCvItems}

    % ---
    % 其它方面
    %\newPartBar{\faHeart}{其他方面}
    % 其它方面的内容
    %\begin{myCvItems}
    %        \item 吃饭
    %        \item 吃饭
    %        \item 吃饭
    %        \item 吃饭
    %\end{myCvItems}
    % ---
\end{document}
